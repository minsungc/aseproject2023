%%
%% This is file `sample-authordraft.tex',
%% generated with the docstrip utility.
%%
%% The original source files were:
%%
%% samples.dtx  (with options: `authordraft')
%% 
%% IMPORTANT NOTICE:
%% 
%% For the copyright see the source file.
%% 
%% Any modified versions of this file must be renamed
%% with new filenames distinct from sample-authordraft.tex.
%% 
%% For distribution of the original source see the terms
%% for copying and modification in the file samples.dtx.
%% 
%% This generated file may be distributed as long as the
%% original source files, as listed above, are part of the
%% same distribution. (The sources need not necessarily be
%% in the same archive or directory.)
%%
%% The first command in your LaTeX source must be the \documentclass command.
\documentclass[sigconf,nonacm]{acmart}
\usepackage{src}
%% NOTE that a single column version may be required for 
%% submission and peer review. This can be done by changing
%% the \doucmentclass[...]{acmart} in this template to 
%% \documentclass[manuscript,screen,review]{acmart}
%% 
%% To ensure 100% compatibility, please check the white list of
%% approved LaTeX packages to be used with the Master Article Template at
%% https://www.acm.org/publications/taps/whitelist-of-latex-packages 
%% before creating your document. The white list page provides 
%% information on how to submit additional LaTeX packages for 
%% review and adoption.
%% Fonts used in the template cannot be substituted; margin 
%% adjustments are not allowed.
%%
%% \BibTeX command to typeset BibTeX logo in the docs
\AtBeginDocument{%
  \providecommand\BibTeX{{%
    \normalfont B\kern-0.5em{\scshape i\kern-0.25em b}\kern-0.8em\TeX}}}

%% Rights management information.  This information is sent to you
%% when you complete the rights form.  These commands have SAMPLE
%% values in them; it is your responsibility as an author to replace
%% the commands and values with those provided to you when you
%% complete the rights form.
\setcopyright{acmcopyright}
\copyrightyear{2023}
\acmYear{2023}
\acmDOI{10.1145/1122445.1122456}

%% These commands are for a PROCEEDINGS abstract or paper.
\acmConference[Woodstock '18]{Woodstock '18: ACM Symposium on Neural
  Gaze Detection}{June 03--05, 2018}{Woodstock, NY}
\acmBooktitle{Woodstock '18: ACM Symposium on Neural Gaze Detection,
  June 03--05, 2018, Woodstock, NY}
\acmPrice{15.00}
\acmISBN{978-1-4503-XXXX-X/18/06}


%%
%% Submission ID.
%% Use this when submitting an article to a sponsored event. You'll
%% receive a unique submission ID from the organizers
%% of the event, and this ID should be used as the parameter to this command.
%%\acmSubmissionID{123-A56-BU3}

%%
%% The majority of ACM publications use numbered citations and
%% references.  The command \citestyle{authoryear} switches to the
%% "author year" style.
%%
%% If you are preparing content for an event
%% sponsored by ACM SIGGRAPH, you must use the "author year" style of
%% citations and references.
%% Uncommenting
%% the next command will enable that style.
%%\citestyle{acmauthoryear}

%%
%% end of the preamble, start of the body of the document source.
\begin{document}

%%
%% The "title" command has an optional parameter,
%% allowing the author to define a "short title" to be used in page headers.
\title{The Evolution of Proof Assistant Math Repositories}

%%
%% The "author" command and its associated commands are used to define
%% the authors and their affiliations.
%% Of note is the shared affiliation of the first two authors, and the
%% "authornote" and "authornotemark" commands
%% used to denote shared contribution to the 

\author{Mahsa Bazzaz}
\affiliation{%
  \institution{Northeastern University}
  \city{Boston}
  \state{Massachusetts}
  \country{USA}
}

\author{Minsung Cho}
\affiliation{%
  \institution{Northeastern University}
  \city{Boston}
  \state{Massachusetts}
  \country{USA}
}

\author{Gwen Lincroft}
\affiliation{%
\institution{Northeastern University}
\city{Boston}
\state{Massachusetts}
\country{USA}}

%%
%% By default, the full list of authors will be used in the page
%% headers. Often, this list is too long, and will overlap
%% other information printed in the page headers. This command allows
%% the author to define a more concise list
%% of authors' names for this purpose.
\renewcommand{\shortauthors}{Bazzaz, Cho, and Lincroft}

%%
%% This command processes the author and affiliation and title
%% information and builds the first part of the formatted document.
\maketitle

\section{Introduction}

Proof assistants (also known as interactive theorem provers) are programming languages with the primary purpose of providing computational meaning to mathematical proof through code. The first proof assistants, developed over fifty years ago, were designed to help reason about the correctness of software and hardware through mathematical abstraction. However, as proof assistants became more expressive and user-friendly, they have gained traction in numerous fields, including artificial intelligence, computer science education, and mathematics.

Over the past 20 years, proof assistants have gained a large amount of attention within the mathematics community with their potential to bridge software development and mathematical proof. This potential has been realized through multiple open-source, well-documented libraries of formalized proofs as programs. Although there have been many such libraries over time, the predominant ones today are written in three languages: Lean, Coq, and Isabelle/HOL.

\textbf{Lean} has been the focal point of math library development in the past five years, gaining the attention of prominent mathematicians by demonstrating its ability to prove research-level mathematics \TODO{add citation.}. Its math library, known as \mathlib, has been open-sourced on GitHub under an Apache license. It has extremely rigid conventions for contributing, including guidelines on style, naming, commit messages, and pull request labeling. 

\textbf{Coq} is one of the most widely-used proof assistants in computer science today. However, its math library is not centralized like \mathlib. Instead, it offers a \textit{loosely federated} list of open-source GitHub repositories. Many of these repositories are under the same GitHub organizations, for example \texttt{math-comp} or \texttt{coq-community}, but not all of them are and their interdependence is not obvious.

\textbf{Isabelle/HOL} is another major proof assistant that offers an extremely strong general automation, which the other proof assistants lack. Although its math library is open source, it is not hosted natively on GitHub and external contribution is not done through GitHub's standard means. Furthermore, there is no centralized repository \'a la \mathlib and most are hosted on the website \textit{Archive of Formal Proofs}, where the proofs lack version control but give a detailed account of authors, dependencies, and proof techniques.

It is clear that each of these proof assistants and their math libraries are different in development. However, they all attempt to demonstrate the power of proof assistants in formalizing known mathematics through code. This project aims to compare and contrast the evolution of these three main repositories through a software engineering lens. In particular, for this project we have two main research questions:

\begin{enumerate}
    \item What mathematical theorems did different formalized mathematics libraries prove? How has this developed over time? How have the proofs changed?
    \item How has the popularity of contributing to theorem provers over time changed? Are there any factors we need to take into account to evaluate proof assistant software different than traditional software?
\end{enumerate}
\section{Project Results}

\section{Project Challenges}

\end{document}